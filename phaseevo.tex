%
% File: pulsedopo/base/pulsedopo.tex [plain TeX code]
% Last change: February 18, 2026
%
% Derivation of the phase evolution in parametric amplification processes
% in chiral media.
%
% Copyright (C) 2026, Fredrik Jonsson, under GPL 3.0. See enclosed LICENSE.
%
\input macros/epsf.tex
\input macros/eplain.tex
\font\ninerm=cmr9
\font\twentyrm=cmr12 at 20 truept
\font\twelvesc=cmcsc10 at 12 truept
\input amssym % to get the {\Bbb E} font (strikethrough E)
\def\document #1 {\hsize=150mm\hoffset=4.6mm\vsize=230mm\voffset=7mm
  \topskip=0pt\baselineskip=12pt\parskip=0pt\leftskip=0pt\parindent=15pt
  \headline={\ifnum\pageno>1\ifodd\pageno\rightheadline\else\leftheadline\fi
    \else\hfill\fi}
  \def\rightheadline{\tenrm{\it #1}
    \hfil{\it\date}}
  \def\leftheadline{\tenrm{\it\date}
    \hfil{\it #1}}
  \noindent~\vskip-60pt\hskip-40pt{\epsfbox{macros/UU_logo_color.eps}}
  \vskip-42pt\hfill\vbox{\hbox{{\it\author}}
  \hbox{{\it\date}}}\vskip 36pt
  \centerline{\twelvesc #1}
  \vskip 24pt\noindent}
\def\section #1 {\bigskip\goodbreak\noindent{\bf #1}
  \par\nobreak\smallskip\noindent}
\def\subsection #1 {\bigskip\goodbreak\noindent{\it #1}
  \par\nobreak\smallskip\noindent}
\def\iint{\mathop{\int\kern-8pt\int}}
\def\iiint{\mathop{\int\kern-8pt\int\kern-8pt\int}}
\def\oiint{\mathop{\int\kern-8pt\int\kern-13.2pt{\bigcirc}}}
\def\Re{\mathop{\rm Re}\nolimits} % real part
\def\Im{\mathop{\rm Im}\nolimits} % imaginary part
\def\Tr{\mathop{\rm Tr}\nolimits} % quantum mechanical trace
\def\fourier{\mathop{\frak F}\nolimits}
\def\eqq{\mathop{\vbox{\hbox{\hskip2pt?}\vskip-6pt\hbox{=}}}}
\def\sn{\mathop{\rm sn}\nolimits}
\def\cn{\mathop{\rm cn}\nolimits}
\def\sech{\mathop{\rm sech}\nolimits}
\def\acosh{\mathop{\rm acosh}\nolimits}

%
% Blackboard bold fonts.
%
\font\mbb=msbm10 
\newfam\bbb
\textfont\bbb=\mbb

%
% Define in which way we want displayed equation numbers to appear in
% the text. In this case, it is preferred to have the equation numbers
% as one single running index, rather than the form otherwise commonly
% found format used in books, (<chapter number>.<equation number>).
%
\def\eqconstruct#1{#1}

%
% Define in which way we want displayed subequation numbers
% to appear in the text.
%
\newcount\subref
\def\eqsubreftext#1#2{%
  \subref = #2           % The space stops a <number>.
  \advance\subref by 96  % `a' is character code 97.
  #1{\rm\char\subref}%
}

%
% Define the 'boxit' macro from D.E. Knuths "The TeXbook, Exercise 21.3.
%
\def\boxit#1{\vbox{\hrule\hbox{\vrule\kern3pt
  \vbox{\kern3pt#1\kern3pt}\kern3pt\vrule}\hrule}}

\def\date{February 18, 2026}
\def\author{Fredrik Jonsson}
\document{Phase evolution in parametric processes in chiral media}
\vskip24pt

\section{Definition of fields and parameters}
The quasi-monochromatic optical field is in the circularly polarized basis, with
complex-valued basis vectors ${\bf e}_+=({\bf e}_x+i{\bf e}_y)/\sqrt{2}$ for left
circular polarization (LCP) and  ${\bf e}_-=({\bf e}_x-i{\bf e}_y)/\sqrt{2}$ for
right circular polarization (RCP), expressed as
$$
  {\bf E}(z,t)=\sum^3_{k=1}{\rm Re}[
      ({\bf e}_+ E^+_{\omega_k}(z)+{\bf e}_-E^-_{\omega_k}(z))\exp(-i\omega_k t)],
  \eqdef{eq:w10}
$$
with $k=1,2,3$ designating idler, signal and pump, respectively.
For the present analysis, we choose a crystal of point-symmetry group 32
(trigonal) is chosen. As this group possess no centre of inversion, the axial
tensor $\gamma_{ijkl}$ for nonlocal interaction shares the same nonzero and
independent elements as the third-order polar tensor $\chi_{ijkl}$ for local
(electric dipolar) interactions, and the electric polarization density of the
medium becomes
$$
  \eqalignno{
  P^{\pm}_{\omega_{1}}
    &=\varepsilon_0\Big(n^2_{1}-1
  \pm i\gamma_{1}{{\partial}\over{\partial z}}\Big)
        E^{\pm}_{\omega_{1}}+\varepsilon_0
      \Big(p_{1} \pm i q_{1}{{\partial}\over{\partial z}}\Big)
       (E^{\mp}_{\omega_3} E^{\pm*}_{\omega_{2}}),\eqdefn{eq:w20}&\eqsubdef{eq:w20a}\cr
  P^{\pm}_{\omega_{2}}
    &=\varepsilon_0\Big(n^2_{2}-1
  \pm i\gamma_{2}{{\partial}\over{\partial z}}\Big)
        E^{\pm}_{\omega_{2}}+\varepsilon_0
      \Big(p_{2} \pm i q_{2}{{\partial}\over{\partial z}}\Big)
       (E^{\mp}_{\omega_3} E^{\pm*}_{\omega_{1}}),&\eqsubdef{eq:w20b}\cr
  P^{\mp}_{\omega_3}
    &=\varepsilon_0\Big(n^2_3-1
  \mp i\gamma_3{{\partial}\over{\partial z}}\Big)
        E^{\mp}_{\omega_3}+\varepsilon_0
      \Big(p_3 \mp i q_3{{\partial}\over{\partial z}}\Big)
        (E^{\pm}_{\omega_1} E^{\pm}_{\omega_2}),&\eqsubdef{eq:w20c}\cr
        }
$$
where the elements of the local ($\chi_{ij}$, $\chi_{ijk}$) and non-local
($\gamma_{ijk}$, $\gamma_{ijkl}$) susceptibility tensors in standard
notation are expressed as
$$
  \eqalign{
    n^2_k&=1+\chi_{xx}(-\omega_k;\omega_k),\quad
    \gamma_k=\gamma_{xyz}(-\omega_k;\omega_k),\cr
    p_{1,2}&=2^{1/2}\chi_{xxx}(-\omega_{1,2};\omega_3,-\omega_{2,1}),\cr
    p_3&=2^{1/2}\chi_{xxx}(-\omega_3;\omega_1,\omega_2),\cr
    q_{1,2}&=2^{1/2}\gamma_{xxyz}(-\omega_{1,2};\omega_3,-\omega_{2,1}),\cr
    q_3&=2^{1/2}\gamma_{xxyz}(-\omega_3;\omega_1,\omega_2).\cr
  }
\eqdef{eq:w30}
$$

\section{Formulation of the equations of motion}
By separating the fields into their forward and backward traveling components,
keeping in mind that the backward traveling LCP and RCP components are to be
connected with their respective complex-conjugated basis vectors ${\bf e}^*_+$
and ${\bf e}^*_-$, respectively, according to
$$
  \eqalign{
    E^{\pm}_{\omega_k}=&a^{f\pm}_{k}\exp(i(k_k\mp\alpha_k)z)
      +a^{b\mp}_{k}\exp(-i(k_k\pm \alpha_k)z)
  }
\eqdef{eq:w40}
$$
with $k_k=\omega_k n_k/c$ and $\alpha_k=k^2_k\gamma_k/2n^2_k$, followed by
averaging over a few spatial periods, the resulting equations for the field
envelopes become, for the forward traveling wave
$$
  \eqalignno{
    {{\partial a^{f\pm}_{1}}\over{\partial z}}
      &=i\kappa^{\pm}_{1} a^{f\mp}_{3} a^{f\pm*}_{2}
          \exp(i(\Delta k\pm\Delta\alpha)z),
    \eqdefn{eq:w50}&\eqsubdef{eq:w50a}\cr
    {{\partial a^{f\pm}_{2}}\over{\partial z}}
      &=i\kappa^{\pm}_{2} a^{f\mp}_{3} a^{f\pm*}_{1}
          \exp(i(\Delta k\pm\Delta\alpha)z),
    &\eqsubdef{eq:w50b}\cr
    {{\partial a^{f\mp}_{3}}\over{\partial z}}
      &=i\kappa^{\mp}_3 a^{f\pm}_{1} a^{f\pm}_{2}
          \exp(-i(\Delta k\pm\Delta\alpha)z),
    &\eqsubdef{eq:w50c}\cr
  }
$$
and for the backward traveling wave
$$
  \eqalignno{
    {{\partial a^{b\mp}_{1}}\over{\partial z}}
      &=-i\kappa^{\mp}_{1} a^{b\pm}_{3} a^{b\mp*}_{2}
           \exp(-i(\Delta k\mp\Delta\alpha)z),
    \eqdefn{eq:w52}&\eqsubdef{eq:w52a}\cr
    {{\partial a^{b\mp}_{2}}\over{\partial z}}
      &=-i\kappa^{\mp}_{2} a^{b\pm}_{3} a^{b\mp*}_{1}
           \exp(-i(\Delta k\mp\Delta\alpha)z),
    &\eqsubdef{eq:w52b}\cr
    {{\partial a^{b\pm}_{3}}\over{\partial z}}
      &=-i\kappa^{\pm}_3 a^{b\mp}_{1} a^{b\mp}_{2}
           \exp(i(\Delta k\mp\Delta\alpha)z),
    &\eqsubdef{eq:w52c}\cr
  }
$$
where $\Delta\alpha=k_3-k_2-k_1$ is the electric dipolar phase mismatch and
$\Delta\alpha=\alpha_1+\alpha_2+\alpha_3$ its nonlocal correction.
In Eqs.~\eqref{eq:w50} and~\eqref{eq:w52}, the coupling coefficients include
the nonlocal interaction between light and matter via the coefficients
$\alpha_j$ and $q_j$ as
$$
  \eqalignno{
  \kappa^{\pm}_1&=i{{k_1}\over{2n^2_1}}[p_1-q_1(\alpha_3+\alpha_2)\mp q_1(k_3-k_2)],
    \eqdefn{eq:w60}&\eqsubdef{eq:w60a}\cr
  \kappa^{\pm}_2&=i{{k_2}\over{2n^2_2}}[p_2-q_2(\alpha_3+\alpha_1)\mp q_2(k_3-k_1)],
    &\eqsubdef{eq:w60b}\cr
  \kappa^{\pm}_3&=i{{k_3}\over{2n^2_3}}[p_3-q_3(\alpha_1+\alpha_2)\mp q_3(k_1+k_2)].
    &\eqsubdef{eq:w60c}\cr
  }
$$

\section{Evolution of field envelopes}
By substitution into real-valued intensity and phase variables,
$$
  \eqalignno{
  a^{f\pm}_k(z)&=\big(\kappa^{\pm}_k u^{\pm}_k(z)\big)^{1/2}\exp(i\varphi^{\pm}_k(z))
  \eqdefn{eq:w70}&\eqsubdef{eq:w70a},\cr
  a^{b\pm}_k(z)&=\big(\kappa^{\pm}_k v^{\pm}_k(z)\big)^{1/2}\exp(i\psi^{\pm}_k(z))
  &\eqsubdef{eq:w70b},\cr
  }
$$
the system~\eqref{eq:w50} for the field envelopes of the forward traveling wave
can by separation into real and imaginary parts be cast into a real-valued and
normalized form. For the forward traveling components this becomes
$$
  \eqalignno{
    {{\partial u^{\pm}_1}\over{\partial z}}
        &={{\partial u^{\pm}_2}\over{\partial z}}
         =-{{\partial u^{\mp}_3}\over{\partial z}}
         =-2\kappa_{\pm}(u^{\pm}_1 u^{\pm}_2 u^{\mp}_3)^{1/2}\sin\varphi_{\pm},
    \eqdefn{eq:w80}&\eqsubdef{eq:w80a}\cr
    {{\partial\varphi_{\pm}}\over{\partial z}}
        &={{1}\over{2}}\cot\varphi_{\pm}{{\partial}\over{\partial z}}
           \ln(u^{\pm}_1 u^{\pm}_2 u^{\mp}_3) + (\Delta k\pm\Delta\alpha),
    &\eqsubdef{eq:w80b}\cr
  }
$$
while for the backward traveling components
$$
  \eqalignno{
    {{\partial v^{\pm}_1}\over{\partial z}}
        &={{\partial v^{\pm}_2}\over{\partial z}}
         =-{{\partial v^{\mp}_3}\over{\partial z}}
         =2\kappa_{\pm}(v^{\pm}_1 v^{\pm}_2 v^{\mp}_3)^{1/2}\sin\psi_{\pm},
    \eqdefn{eq:w90}&\eqsubdef{eq:w90a}\cr
    {{\partial\psi_{\pm}}\over{\partial z}}
        &={{1}\over{2}}\cot\psi_{\pm}{{\partial}\over{\partial z}}
           \ln(v^{\pm}_1 v^{\pm}_2 v^{\mp}_3) - (\Delta k\pm\Delta\alpha),
    &\eqsubdef{eq:w90b}\cr
  }
$$
with common coupling coefficients
$$
  \kappa_{\pm}=(\kappa^{\pm}_1 \kappa^{\pm}_2 \kappa^{\mp}_3)^{1/2}
  \eqdef{eq:w92}
$$
and with the relative phase angles $\varphi_{\pm}(z)$ and $\psi_{\pm}(z)$
expressed in terms of the individual phases of the field envelopes as
$$
  \eqalignno{
    \varphi_{\pm}&=\varphi^{\mp}_3-\varphi^{\pm}_2-\varphi^{\pm}_1
       +(\Delta k\pm\Delta\alpha)z,
    \eqdefn{eq:w100}&\eqsubdef{eq:w100a}\cr
    \psi_{\pm}&=\psi^{\mp}_3-\psi^{\pm}_2-\psi^{\pm}_1
       -(\Delta k\pm\Delta\alpha)z.
    &\eqsubdef{eq:w100b}\cr
  }
$$
From Eqs.~\eqref{eq:w80a} and~\eqref{eq:w90a}, which constitute the
Manley--Rowe relations for the energy transfer rate between frequency
components of the counter-propagating waves, one obtains the set of
invariants of motion $\tilde{u}^{\pm}_k$ and $\tilde{v}^{\pm}_k$ determined
from the boundary conditions as
$$
  \eqalignno{
  u^{\pm}_2+u^{\mp}_3&=m^{\pm}_1,\qquad
  u^{\pm}_1+u^{\mp}_3=m^{\pm}_2,\qquad
  u^{\pm}_1-u^{\pm}_2=m^{\mp}_3,
  \eqdefn{eq:w110}&\eqsubdef{eq:w110a}\cr
  v^{\pm}_2+v^{\mp}_3&=m'^{\pm}_1,\qquad
  v^{\pm}_1+v^{\mp}_3=m'^{\pm}_2,\qquad
  v^{\pm}_1-v^{\pm}_2=m'^{\mp}_3,
  &\eqsubdef{eq:w110b}\cr
  }
$$
with constants $m^{\pm}_k$ and $m'^{\pm}_k$. These relations allow us to solve
for, say, the counter-propagating pump $u^{\pm}_3(z)$ and $v^{\pm}_3(z)$, from
which the general idler and signal fields directly can be obtained.

By multiplying Eqs.~\eqref{eq:w80b} and~\eqref{eq:w90b} by
$\sin(\varphi_{\pm})$ and $\sin(\psi_{\pm})$, respectively, followed by
identification of total differentials, these equations can be integrated
for elimination of the relative phases in Eqs.~\eqref{eq:w80a}
and~\eqref{eq:w90a} as
$$
  \eqalignno{
  \cos\varphi_{\pm}&=(\Gamma_{\pm}-\Delta\aleph_{\pm}u^{\mp}_3)
     (u^{\pm}_1 u^{\pm}_2 u^{\mp}_3)^{-1/2}
  \eqdefn{eq:w120}&\eqsubdef{eq:w120a}\cr
  \cos\psi_{\pm}&=(\Gamma'_{\pm}-\Delta\aleph_{\pm}v^{\mp}_3)
     (v^{\pm}_1 v^{\pm}_2 v^{\mp}_3)^{-1/2}
  \eqsubdef{eq:w120b}\cr
  }
$$
where $\Gamma_{\pm}$ and $\Gamma'_{\pm}$ are constants of integration for the
forward and backward traveling LCP/RCP components, and where
$$
  \Delta\aleph_{\pm}={{(\Delta k\pm\Delta\alpha)}\over{2\kappa_{\pm}}}
\eqdef{eq:w130}
$$
is the phase mismatch normalized to the coupling efficiency. This taken together
with the invariants of motion~(\ref{eq:w110}) directly leads to the equations
for the RCP/LCP pump envelopes taking the form
$$
  \eqalignno{
  {{\partial u^{\mp}_3}\over{\partial z}}&=2\kappa_{\pm}\big[
    (m^{\pm}_1-u^{\mp}_3)(m^{\pm}_2-u^{\mp}_3)u^{\mp}_3
      -(\Gamma_{\pm}-\Delta\aleph_{\pm}u^{\mp}_3)^2
  \big]^{1/2},
  \eqdefn{eq:w140}&\eqsubdef{eq:w140a}\cr
  {{\partial v^{\mp}_3}\over{\partial z}}&=-2\kappa_{\pm}\big[
    (m'^{\pm}_1-v^{\mp}_3)(v'^{\pm}_2-v^{\mp}_3)v^{\mp}_3
      -(\Gamma'_{\pm}-\Delta\aleph_{\pm}v^{\mp}_3)^2
  \big]^{1/2}.
  &\eqsubdef{eq:w140b}\cr
  }
$$
In order to solve Eqs.~\eqref{eq:w140}, the integration constants $\Gamma_{\pm}$
and $\Gamma'_{\pm}$ need to be determined by the boundary conditions of the
amplitudes and phases, typically applied at the entry $z=0$ and exit $z=L$ of
the nonlinear medium and by using the expressions from Eqs.~\eqref{eq:w120}.
This way, the third-order polynomial in the square root of the right-hand sides
may be factorized by Cardano's method to yield the equations
$$
  \eqalignno{
  {{\partial u^{\mp}_3}\over{\partial z}}&=2\kappa_{\pm}\big[
    (u^{\mp}_3-u^{\mp}_{3a})(u^{\mp}_3-u^{\mp}_{3b})(u^{\mp}_3-u^{\mp}_{3c})
  \big]^{1/2},
  \eqdefn{eq:w150}&\eqsubdef{eq:w150a}\cr
  {{\partial v^{\mp}_3}\over{\partial z}}&=-2\kappa_{\pm}\big[
    (v^{\mp}_3-v^{\mp}_{3a})(v^{\mp}_3-v^{\mp}_{3b})(v^{\mp}_3-v^{\mp}_{3c})
  \big]^{1/2},
  \eqsubdef{eq:w150b}\cr
  }
$$
where the real-valued roots for the following discussion without loss of
generality are assumed to obey $u^{\mp}_{3a} \le u^{\mp}_{3c} \le u^{\mp}_{3c}$
and $v^{\mp}_{3a} \le v^{\mp}_{3c} \le v^{\mp}_{3c}$.

By reformulating the field variables $u^{\mp}_3$ and $u^{\mp}_3$ in terms of
temporary expressions of the form
$(u^{\pm}_3-u^{\pm}_{3a})^{1/2}(u^{\pm}_{3b}-u^{\pm}_{3a})^{-1/2}$ and
$(v^{\pm}_3-v^{\pm}_{3a})^{1/2}(v^{\pm}_{3b}-v^{\pm}_{3a})^{-1/2}$, respectively,
Eqs.~\eqref{eq:w150} are expressed in an elliptic integral form which can be
integrated into explicit solutions for the forward and backward traveling pump
waves in terms of Jacobian elliptic functions, as
$$
  \eqalignno{
  u^{\mp}_3(z)=u^{\mp}_{3a}+(u^{\mp}_{3b}-u^{\mp}_{3a})
  \sn^2&[\kappa_{\pm}(u^{\mp}_{3c}-u^{\mp}_{3a})^{1/2}z
                      +F(\sigma_{\pm},\xi_{\pm}),\xi_{\pm}],
  \eqdefn{eq:w160}&\eqsubdef{eq:w160a}\cr
  v^{\mp}_3(z)=v^{\mp}_{3a}+(v^{\mp}_{3b}-v^{\mp}_{3a})
  \sn^2&[\kappa_{\pm}(v^{\mp}_{3c}-v^{\mp}_{3a})^{1/2}(L-z)
                      +F(\sigma'_{\pm},\xi'_{\pm}),\xi'_{\pm}],
  \qquad
  &\eqsubdef{eq:w160b}\cr
  }
$$
where $F(\sigma,\xi)$ denotes the normal elliptic integral of the first
kind\numberedfootnote{P.~F. Byrd and M.~D. Friedman, {\it Handbook of
  Elliptic Integrals for Engineers and Scientists} (Springer, 1971).}
$$
  F(\sigma,\xi)=\int^{\sigma}_0{{ds}\over{[(1-s^2)(1-\xi^2 s^2)]^{1/2}}}
\eqdef{eq:w170}
$$
with moduli for the forward and backward traveling wave envelopes
$$
  \xi_{\pm}=\bigg(
    {{u^{\mp}_{3b}-u^{\mp}_{3a}}\over{u^{\mp}_{3c}-u^{\mp}_{3a}}}
  \bigg)^{1/2},\quad
  \xi'_{\pm}=\bigg(
    {{v^{\mp}_{3b}-v^{\mp}_{3a}}\over{v^{\mp}_{3c}-v^{\mp}_{3a}}}
  \bigg)^{1/2},
\eqdef{eq:w180}
$$
and with respective upper limits of integration
$$
  \eqalignno{
  \sigma_{\pm}&=[(u^{\mp}_3(0)-u^{\mp}_{3a})/(u^{\mp}_{3b}-u^{\mp}_{3a})]^{1/2},
  \eqdefn{eq:w190}&\eqsubdef{eq:w190a}\cr
  \sigma'_{\pm}&=[(v^{\mp}_3(L)-v^{\mp}_{3a})/(v^{\mp}_{3b}-v^{\mp}_{3a})]^{1/2}.
  &\eqsubdef{eq:w190b}\cr
  }
$$
The general solutions in Eqs.~\eqref{eq:w160} for the forward and backward
traveling pump waves directly provides the explicit solutions for the idler
and signal waves via the invariants of motion from Eqs.~\eqref{eq:w110}.

\section{Solutions for the envelopes for singly resonant parametric oscillation}
For the present purpose, we focus on a singly resonant configuration with only
the signal being reflected. Also, we assume that no idler wave is present at
$z=0$, hence with $m^{\pm}_1=u^{\pm}_2(0)+u^{\mp}_3(0)$, $m^{\pm}_2=u^{\mp}_3(0)
$ and
$m^{\mp}_3=-u^{\pm}_2(0)$, leading to the solution for the forward propagating
phase difference as
$$
  \cos\varphi_{\pm}={{(u^{\mp}_3(0)-u^{\mp}_3)\Delta\aleph_{\pm}}
                        \over{(u^{\pm}_1 u^{\pm}_2 u^{\mp}_3)^{1/2}}},
\eqdef{eq:o10}
$$
where as previously $\Delta\aleph_{\pm}=(\Delta k\pm\Delta\alpha)/(2\kappa_{\pm})$
is the phase mismatch normalized to the coupling efficiency.
The general solution for the fields of the parametric process is expressed in
Jacobian elliptic functions\numberedfootnote{P.~F. Byrd and M.~D. Friedman,
  {\it Handbook of Elliptic Integrals for Engineers and Scientists}
  (Springer, 1971).}
as
$$
  \eqalignno{
    \hbox{Idler:}\qquad
      u^{\pm}_1(z)&=u^{\mp}_3(0)-u^{\mp}_3(z),
    \eqdefn{eq:o20}&\eqsubdef{eq:o20a}\cr
    \hbox{Signal:}\qquad
      u^{\pm}_2(z)&=u^{\pm}_2(0)+u^{\mp}_3(0)-u^{\mp}_3(z),
    &\eqsubdef{eq:o20b}\cr
    \hbox{Pump:}\qquad
      u^{\mp}_3(z)&=u^{\mp}_{3a}+(u^{\mp}_{3b}-u^{\mp}_{3a})
        \sn^2\big(\kappa_{\pm}(u^{\mp}_{3c}-u^{\mp}_{3a})^{1/2}z
               +K(\xi_{\pm}),\xi_{\pm}\big),
    &\eqsubdef{eq:o20c}\cr
  }
$$
where the constant coefficients are
$$
  \eqalignno{
  u^{\mp}_{3a}&={{u^{\mp}_{3}(0)}\over{2}}\big\{1+s^2_{\pm}+\phi^2_{\pm}
      -\big[(1+s^2_{\pm}+\phi^2_{\pm})^2-4\phi^2_{\pm}\big]^{1/2}\big\}
  \eqdefn{eq:o30}&\eqsubdef{eq:o30a}\cr
  u^{\mp}_{3b}&=u^{\mp}_{3}(0)
  &\eqsubdef{eq:o30b}\cr
  u^{\mp}_{3c}&={{u^{\mp}_{3}(0)}\over{2}}\big\{1+s^2_{\pm}+\phi^2_{\pm}
      +\big[(1+s^2_{\pm}+\phi^2_{\pm})^2-4\phi^2_{\pm}\big]^{1/2}\big\}
  &\eqsubdef{eq:o30c}\cr
  }
$$
with
$$
  s^2_{\pm}=u^{\pm}_2(0)/u^{\mp}_3(0)
  \eqdef{eq:o36}
$$
as the normalized initial signal vs pump intensity,
$$
  \phi^2_{\pm}=\Delta\aleph^2_{\pm}/u^{\mp}_3(0)
\eqdef{eq:o38}
$$
the phase mismatch normalized against the pump intensity, and where
$$
  K(\xi)=\int^1_0{{ds}\over{[(1-s^2)(1-\xi^2 s^2)]^{1/2}}}
\eqdef{eq:o40}
$$
is the complete elliptic integral of the first kind in the
standard form, with modulus
$$
  \xi_{\pm}=\bigg({{u^{\mp}_{3b}-u^{\mp}_{3a}}
                  \over{u^{\mp}_{3c}-u^{\mp}_{3a}}}\bigg)^{1/2}.
\eqdef{eq:o50}
$$

\section{Solving for the phases of the envelopes}
By inserting the expressions for the complex-valued envelopes $a^{f\pm}_k$ from
Eq.~\eqref{eq:w70a} into Eqs.~\eqref{eq:w50a}--\eqref{eq:w50c}, and taking
the imaginary part, we obtain the equations of motion for the phases
$\phi^{\pm}_k$ as (compare with Eqs.~\eqref{eq:w80})\numberedfootnote{Recall
  that the common phase difference $\varphi_{\pm}$ in the parametric interaction
  is defined by Eq.~\eqref{eq:w100a} as
  $$
    \varphi_{\pm}\equiv\varphi^{\mp}_3-\varphi^{\pm}_2-\varphi^{\pm}_1
       +(\Delta k\pm\Delta\alpha)z,
  $$}
$$
  \eqalignno{
    \hbox{Idler:}\qquad
    {{\partial\varphi^{\pm}_1(z)}\over{\partial z}}
      &=\kappa_{\pm}
          \bigg({{u^{\mp}_3(z) u^{\pm}_2(z)}\over{u^{\pm}_1(z)}}\bigg)^{1/2}
          \cos\varphi_{\pm}(z),
    \eqdefn{eq:ph10}&\eqsubdef{eq:ph10a}\cr
    \hbox{Signal:}\qquad
    {{\partial\varphi^{\pm}_2(z)}\over{\partial z}}
      &=\kappa_{\pm}
          \bigg({{u^{\mp}_3(z) u^{\pm}_1(z)}\over{u^{\pm}_2(z)}}\bigg)^{1/2}
          \cos\varphi_{\pm}(z),
    &\eqsubdef{eq:ph10b}\cr
    \hbox{Pump:}\qquad
    {{\partial\varphi^{\mp}_3(z)}\over{\partial z}}
      &=-\kappa_{\pm}
          \bigg({{u^{\pm}_1(z) u^{\pm}_2(z)}\over{u^{\mp}_3(z)}}\bigg)^{1/2}
          \cos\varphi_{\pm}(z),
    &\eqsubdef{eq:ph10c}\cr
  }
$$
with the common coupling coefficient $\kappa_{\pm}$ as previously given by
Eq.~\eqref{eq:w92}. We can here directly substitute for the $\cos\varphi_{\pm}$
factor by using Eq.~\eqref{eq:o10},
$$
  \cos\varphi_{\pm}={{(u^{\mp}_3(0)-u^{\mp}_3)\Delta\aleph_{\pm}}
                        \over{(u^{\pm}_1 u^{\pm}_2 u^{\mp}_3)^{1/2}}},
$$
from which we obtain Eqs.~\eqref{eq:ph10} as
$$
  \eqalignno{
    \hbox{Idler:}\qquad
    {{\partial\varphi^{\pm}_1(z)}\over{\partial z}}
      &=\kappa_{\pm}
          {{\big(u^{\mp}_3(0) - u^{\mp}_3(z)\big)}
            \over{u^{\pm}_1(z)}}\Delta\aleph_{\pm},
    \eqdefn{eq:ph20}&\eqsubdef{eq:ph20a}\cr
    \hbox{Signal:}\qquad
    {{\partial\varphi^{\pm}_2(z)}\over{\partial z}}
      &=\kappa_{\pm}
          {{\big(u^{\mp}_3(0) - u^{\mp}_3(z)\big)}
            \over{u^{\pm}_2(z)}}\Delta\aleph_{\pm},
    &\eqsubdef{eq:ph20b}\cr
    \hbox{Pump:}\qquad
    {{\partial\varphi^{\mp}_3(z)}\over{\partial z}}
      &=-\kappa_{\pm}
          {{\big(u^{\mp}_3(0) - u^{\mp}_3(z)\big)}
            \over{u^{\mp}_3(z)}}\Delta\aleph_{\pm}.
    &\eqsubdef{eq:ph20c}\cr
  }
$$
It is convenient to cast this form into a normalized form, in which the
amplitudes are normalized against the initial value of the pump amplitude.
Also, we make use of the definition of the normalized phase variable
$\phi_{\pm}$, from Eq.~\eqref{eq:o38}\numberedfootnote{We define $\phi_{\pm}$ as
  $$
    \phi^2_{\pm}=\Delta\aleph^2_{\pm}/u^{\mp}_3(0)\quad
      \Leftrightarrow\quad\phi_{\pm}=\Delta\aleph_{\pm}/(u^{\mp}_3(0))^{1/2}.
  $$}
as
$$
  \eqalignno{
    \hbox{Idler:}\qquad
    {{\partial\varphi^{\pm}_1(z)}\over{\partial z}}
      &=\underbrace{
          \kappa_{\pm}\big(u^{\mp}_3(0)\big)^{1/2}
          }_{\equiv\zeta_{\pm}/L}
          {{\big(1 - u^{\mp}_3(z)/u^{\mp}_3(0)\big)}
            \over{(u^{\pm}_1(z)/u^{\mp}_3(0))}}
          \underbrace{
            {{\Delta\aleph_{\pm}}\over{\big(u^{\mp}_3(0)\big)^{1/2}}}
          }_{\equiv\phi_{\pm}}\cr
      &=\Big({{\zeta_{\pm}\phi_{\pm}}\over{L}}\Big)
          {{\big(1 - u^{\mp}_3(z)/u^{\mp}_3(0)\big)}
            \over{(u^{\pm}_1(z)/u^{\mp}_3(0))}},
    \eqdefn{eq:ph30}&\eqsubdef{eq:ph30a}\cr
    \hbox{Signal:}\qquad
    {{\partial\varphi^{\pm}_2(z)}\over{\partial z}}
      &=\Big({{\zeta_{\pm}\phi_{\pm}}\over{L}}\Big)
          {{\big(1 - u^{\mp}_3(z)/u^{\mp}_3(0)\big)}
            \over{(u^{\pm}_2(z)/u^{\mp}_3(0))}},
    &\eqsubdef{eq:ph30b}\cr
    \hbox{Pump:}\qquad
    {{\partial\varphi^{\mp}_3(z)}\over{\partial z}}
      &=-\Big({{\zeta_{\pm}\phi_{\pm}}\over{L}}\Big)
          {{\big(1 - u^{\mp}_3(z)/u^{\mp}_3(0)\big)}
            \over{(u^{\mp}_3(z)/u^{\mp}_3(0))}}.
    &\eqsubdef{eq:ph30c}\cr
  }
$$
All occurrences\numberedfootnote{Keeping in mind the convention that $k=1$
  corresponds to the idler, $k=2$ to the signal, and $k=3$ to the pump.}
of $u^{\pm}_k$ in the right-hand sides of Eqs.~\eqref{eq:ph30} all are known
functions, given explicitly by Eqs.~\eqref{eq:o20} with coefficients from
Eqs.~\eqref{eq:o30}.

\section{Solving the phase of the pump wave}
If we start with the phase $\varphi^{\mp}_3$ for the pump, inserting the explicit
solution given by Eq.~\eqref{eq:o20c} into Eq.~\eqref{eq:ph30c}, we after some
cumbersome but straightforward algebra obtain Eq.~\eqref{eq:ph30c} in the form
$$
    {{\partial\varphi^{\mp}_3(z)}\over{\partial z}}
       =-C_{\pm}\bigg(
         {{1-A_{\pm}\sn^2\big(D_{\pm}z+K(\xi_{\pm}),\xi_{\pm}\big)}
           \over{1+B_{\pm}\sn^2\big(D_{\pm}z+K(\xi_{\pm}),\xi_{\pm}\big)}}
       \bigg),
  \eqdef{eq:ph40}
$$
with solution given by the integral
$$
  \varphi^{\mp}_3(z)-\varphi^{\mp}_3(0)
       =-C_{\pm}\int^z_0
       \bigg(
         {{1-A_{\pm}\sn^2\big(D_{\pm}z'+K(\xi_{\pm}),\xi_{\pm}\big)}
           \over{1+B_{\pm}\sn^2\big(D_{\pm}z'+K(\xi_{\pm}),\xi_{\pm}\big)}}
       \bigg)\,dz',
  \eqdef{eq:ph50}
$$
where the constant coefficients $A_{\pm}$, $B_{\pm}$, $C_{\pm}$ and $D_{\pm}$ are
given as
$$
  \eqalignno{
    A_{\pm}&={{(u^{\mp}_{3b}-u^{\mp}_{3a})/u^{\mp}_3(0)}
              \over{{\big(1-u^{\mp}_{3a}/u^{\mp}_3(0)\big)}}},
    \eqdefn{eq:ph60}&\eqsubdef{eq:ph60a}\cr
    B_{\pm}&={{(u^{\mp}_{3b}-u^{\mp}_{3a})/u^{\mp}_3(0)}
              \over{{\big(u^{\mp}_{3a}/u^{\mp}_3(0)\big)}}},
    &\eqsubdef{eq:ph60b}\cr
    C_{\pm}&=\Big({{\zeta_{\pm}\phi_{\pm}}\over{L}}\Big)
            {{\big(1-u^{\mp}_{3a}/u^{\mp}_3(0)\big)}
              \over{{\big(u^{\mp}_{3a}/u^{\mp}_3(0)\big)}}},
    &\eqsubdef{eq:ph60c}\cr
    D_{\pm}&=\Big({{\zeta_{\pm}}\over{L}}\Big)
            \bigg({{u^{\mp}_{3c}-u^{\mp}_{3a}}
              \over{{u^{\mp}_3(0)}}}\bigg)^{1/2},
    &\eqsubdef{eq:ph60d}\cr
  }
$$
where $u^{\mp}_{3a}$, $u^{\mp}_{3b}$ and $u^{\mp}_{3c}$ are given explicitly by
Eqs.~\eqref{eq:o30}.

\section{Solving the phase integral involving Jacobi elliptic functions}
In other words, we have now arrived at the point where we from
Eq.~\eqref{eq:ph50} need to evaluate the integral of the form
$$
  I=\int^z_0{{1-A\sn^2\big(Dz'+K(\xi),\xi\big)}
           \over{1+B\sn^2\big(Dz'+K(\xi),\xi\big)}}\,dz'.
  \eqdef{eq:ph70}
$$
We first substitute
$$
  u=Dz'+K(\xi)\quad\Leftrightarrow\quad dz'=du/D,
  \eqdef{eq:ph80}
$$
which gives
$$
  I={{1}\over{D}}\int^{Dz+K(\xi)}_{K(\xi)}{{1-A\sn^2\big(u,\xi\big)}
           \over{1+B\sn^2\big(u,\xi\big)}}\,du.
  \eqdef{eq:ph90}
$$
We may now split the integrand by partial fraction decomposition%
\numberedfootnote{Swedish {\it partialbr{\aa}ksuppdelning}; you may
  easily verify that this relation holds.}
$$
  {{1-A\sn^2\big(u,\xi\big)}\over{1+B\sn^2\big(u,\xi\big)}}
    =-{{A}\over{B}}+{{(A+B)}\over{B}}{{1}\over{1+B\sn^2\big(u,\xi\big)}},
  \eqdef{eq:ph100}
$$
which enables us to split the integral into two easier terms, of which the
first one is trivial, as
$$
  \eqalign{
    I&={{1}\over{D}}\int^{Dz+K(\xi)}_{K(\xi)}
          \bigg(-{{A}\over{B}}
                +{{(A+B)}\over{B}}{{1}\over{1+B\sn^2\big(u,\xi\big)}}
          \bigg)\,du\cr
     &=-{{A}\over{BD}}\int^{Dz+K(\xi)}_{K(\xi)}\,du
          +{{(A+B)}\over{BD}}\int^{Dz+K(\xi)}_{K(\xi)}
                {{du}\over{1+B\sn^2\big(u,\xi\big)}}\cr
     &=-{{Az}/{B}}
          +{{(A+B)}\over{BD}}\int^{Dz+K(\xi)}_{K(\xi)}
                {{du}\over{1+B\sn^2\big(u,\xi\big)}}\cr
  }
  \eqdef{eq:ph110}
$$
As for the second remaining integral, we may here identify the standard
identity\numberedfootnote{The incomplete elliptic integral of the third
  kind is defined in terms of the Jacobi elliptic function $\sn(u,\xi)$ as
  $$
    \Pi(u,\alpha^2)=\int^{u}_0{{du'}\over{1-\alpha^2\sn^2(u',\xi)}}
  $$
  in the Legendre canonical form and in the notation of Byrd and Friedman,
  only with the difference that we here adopt to the notation $\xi\equiv k$
  as here used. We here consider the elliptic modulus $\xi$ to be implicitly
  present in this integral, via the presence of $\sn(u)=sn(u,\xi)$.
  From its definition, the incomplete elliptic integral of the third kind has
  the property
  $$
    {{d}\over{du}}\Pi(\varphi,\alpha^2,\xi)={{1}\over{1-\alpha^2\sn^2(u,\xi)}}.
  $$
  See P.~F. Byrd and M.~D. Friedman,
  {\it Handbook of Elliptic Integrals for Engineers and Scientists}
  (Springer, 1971), {\bf 400.01}, p.~223.}
for the incomplete elliptic integral of the third kind,
$\Pi(\varphi,\alpha^2,k)$, as
$$
  {{d}\over{du}}\Pi(u,\alpha^2)={{1}\over{1-\alpha^2\sn^2(u,\xi)}},
  \eqdef{eq:ph120}
$$
where we may identify $B=-\alpha^2$. Using this, we may hence cast the integral
of interest into
$$
  \eqalign{
    I&=-{{Az}/{B}}
          +{{(A+B)}\over{BD}}\Big[\Pi(u,-B)\Big]^{Dz+K(\xi)}_{K(\xi)}\cr
     &=-{{Az}/{B}}
          +{{(A+B)}\over{BD}}\big[\Pi(Dz+K(\xi),-B)-\Pi(K(\xi),-B)\big].\cr
  }
  \eqdef{eq:ph130}
$$

\section{Summarizing the solution for the phase of the pump wave}
To summarize, the phase of interest given by Eq.~\eqref{eq:ph50} hence takes
the form
$$
  \varphi^{\mp}_3(z)-\varphi^{\mp}_3(0)
    ={{A_{\pm}C_{\pm}}\over{B_{\pm}}}z
       -{{(A_{\pm}+B_{\pm})C_{\pm}}\over{B_{\pm}D_{\pm}}}
          \big[\Pi(D_{\pm}z+K(\xi_{\pm}),-B_{\pm})-\Pi(K(\xi_{\pm}),-B_{\pm})\big],
  \eqdef{eq:ph140}
$$
with the coefficients $A_{\pm}$, $B_{\pm}$, $C_{\pm}$ and $D_{\pm}$ as previously
given by Eqs.~\eqref{eq:ph60}. To recapitulate the first-principles form of
this solution, this is expressed as
$$
  \varphi^{\mp}_3(z)-\varphi^{\mp}_3(0)
    ={{A_{\pm}C_{\pm}}\over{B_{\pm}}}z
       -{{(A_{\pm}+B_{\pm})C_{\pm}}\over{B_{\pm}D_{\pm}}}
          \int^{D_{\pm}z+K(\xi_{\pm})}_{K(\xi_{\pm})}
            {{du}\over{1+B_{\pm}\sn^2(u,\xi_{\pm})}}.
  \eqdef{eq:ph140}
$$
Depending on the implementation of this integral in some suitable programming
language, we may benefit from reformulating the integral by substituting for
the elliptic function as\numberedfootnote{For example, in the implementation
  of {\tt ellippi(n,phi,m)} in the {\tt mpmath} Python package, the incomplete
  elliptic integral of the third kind is defined in the form
  $$
    \Pi(n;\phi,m)=\int^{\phi}_0{{dt}\over{(1-n\sin^2t)(1-m\sin^2t)^{1/2}}}.
  $$
  See {\tt https://mpmath.org/doc/current/functions/elliptic.html\#ellippi}
  for details.
}
$$
  \eqalign{
  \sn(u,\xi)=\sin(\varphi)
  \qquad&\Leftrightarrow\qquad
  {{d\sn(u,\xi)}\over{du}}du=\cos(\varphi)d\varphi\cr
  \qquad&\Leftrightarrow\qquad
  (1-\sn^2(u,\xi))^{1/2}(1-\xi^2\sn^2(u,\xi))^{1/2}du=\cos(\varphi)d\varphi\cr
  \qquad&\Leftrightarrow\qquad
  (1-\sin^2\varphi)^{1/2}(1-\xi^2\sin^2\varphi)^{1/2}du=\cos(\varphi)d\varphi\cr
  \qquad&\Leftrightarrow\qquad
  du={{\cos(\varphi)}\over{(1-\sin^2\varphi)^{1/2}(1-\xi^2\sin^2\varphi)^{1/2}}}
     d\varphi\cr
  \qquad&\Leftrightarrow\qquad
  du={{d\varphi}\over{(1-\xi^2\sin^2\varphi)^{1/2}}}\cr
  }
  \eqdef{eq:ph150}
$$
which gives
$$
  \varphi^{\mp}_3(z)-\varphi^{\mp}_3(0)
    ={{A_{\pm}C_{\pm}}\over{B_{\pm}}}z
       -{{(A_{\pm}+B_{\pm})C_{\pm}}\over{B_{\pm}D_{\pm}}}
          \int\limits^{\arcsin(\sn(D_{\pm}z+K(\xi_{\pm}),\xi))}_{\arcsin(\sn(K(\xi_{\pm}),\xi))}
{{d\varphi}\over{(1+B_{\pm}\sin^2\varphi)(1-\xi^2\sin^2\varphi)^{1/2}}}.
  \eqdef{eq:ph160}
$$

\section{Solving the phase of the signal wave}
In similar to the procedure for the phase of the pump wave, we for the phase
$\varphi^{\mp}_3$ of the signal wave insert the explicit solution given by
Eq.~\eqref{eq:o20b} into Eq.~\eqref{eq:ph30b}, and after some straightforward
algebra obtain Eq.~\eqref{eq:ph30b} in the form
$$
    {{\partial\varphi^{\pm}_2(z)}\over{\partial z}}
       =C'_{\pm}\bigg(
         {{1-A'_{\pm}\sn^2\big(D'_{\pm}z+K(\xi_{\pm}),\xi_{\pm}\big)}
           \over{1-B'_{\pm}\sn^2\big(D'_{\pm}z+K(\xi_{\pm}),\xi_{\pm}\big)}}
       \bigg),
  \eqdef{eq:ph170}
$$
with solution given by the integral\numberedfootnote{Notice the change of
  sign of $B'_{\pm}$ and $C'_{\pm}$ compared to the similar denominator for
  the phase $\varphi^{\pm}_3$ of the pump wave.}
$$
  \varphi^{\pm}_2(z)-\varphi^{\pm}_2(0)
       =C'_{\pm}\int^z_0
       \bigg(
         {{1-A'_{\pm}\sn^2\big(D'_{\pm}z'+K(\xi_{\pm}),\xi_{\pm}\big)}
           \over{1-B'_{\pm}\sn^2\big(D'_{\pm}z'+K(\xi_{\pm}),\xi_{\pm}\big)}}
       \bigg)\,dz'=C'_{\pm}I,
  \eqdef{eq:ph180}
$$
where the constant coefficients $A'_{\pm}$, $B'_{\pm}$, $C'_{\pm}$ and $D'_{\pm}$
are given as
$$
  \eqalignno{
    A'_{\pm}&={{(u^{\mp}_{3b}-u^{\mp}_{3a})/u^{\mp}_3(0)}
              \over{{\big(1-u^{\mp}_{3a}/u^{\mp}_3(0)\big)}}}=A_{\pm},
    \eqdefn{eq:ph190}&\eqsubdef{eq:ph190a}\cr
    B'_{\pm}&={{(u^{\mp}_{3b}-u^{\mp}_{3a})/u^{\mp}_3(0)}
              \over{{1+s^2_{\pm}-\big(u^{\mp}_{3a}/u^{\mp}_3(0)\big)}}},
    &\eqsubdef{eq:ph190b}\cr
    C'_{\pm}&=\Big({{\zeta_{\pm}\phi_{\pm}}\over{L}}\Big)
            {{\big(1-u^{\mp}_{3a}/u^{\mp}_3(0)\big)}
              \over{{1+s^2_{\pm}-\big(u^{\mp}_{3a}/u^{\mp}_3(0)\big)}}},
    &\eqsubdef{eq:ph190c}\cr
    D'_{\pm}&=\Big({{\zeta_{\pm}}\over{L}}\Big)
            {{(u^{\mp}_{3c}-u^{\mp}_{3a})^{1/2}}
              \over{{u^{\mp}_3(0)}}}=D_{\pm},
    &\eqsubdef{eq:ph190d}\cr
  }
$$
where $u^{\mp}_{3a}$, $u^{\mp}_{3b}$ and $u^{\mp}_{3c}$ as previously are given
explicitly by Eqs.~\eqref{eq:o30}. Just as for the pump, we may now split
the integrand by partial fraction decomposition\numberedfootnote{For the
  signal though with a different sign of $B$, leading to a partial fraction
  decomposition of the form
  $$
    {{1-A\sn^2\big(u,\xi\big)}\over{1-B\sn^2\big(u,\xi\big)}}
      ={{A}\over{B}}-{{(A-B)}\over{B}}{{1}\over{1-B\sn^2\big(u,\xi\big)}},
  $$
  but otherwise split in a completely analogous manner.}
which again enables us to split the integral into two easier terms, of which
the first one is trivial and the second interpreted as an incomplete elliptic
integral of the third kind, as
$$
  \eqalign{
    I&={{1}\over{D}}\int^{Dz+K(\xi)}_{K(\xi)}
          \bigg({{A}\over{B}}
                -{{(A-B)}\over{B}}{{1}\over{1-B\sn^2\big(u,\xi\big)}}
          \bigg)\,du\cr
     &={{A}\over{BD}}\int^{Dz+K(\xi)}_{K(\xi)}\,du
          -{{(A-B)}\over{BD}}\int^{Dz+K(\xi)}_{K(\xi)}
                {{du}\over{1-B\sn^2\big(u,\xi\big)}}\cr
     &={{Az}/{B}}
          -{{(A-B)}\over{BD}}\int^{Dz+K(\xi)}_{K(\xi)}
                {{du}\over{1-B\sn^2\big(u,\xi\big)}}\cr
     &={{Az}/{B}}
          -{{(A-B)}\over{BD}}\Big[\Pi(u,B)\Big]^{Dz+K(\xi)}_{K(\xi)}\cr
     &={{Az}/{B}}
          -{{(A-B)}\over{BD}}\big[\Pi(Dz+K(\xi),B)-\Pi(K(\xi),B)\big].\cr
  }
  \eqdef{eq:ph200}
$$

\section{Summarizing the solution for the phase of the signal wave}
To summarize, the phase of the signal envelope as given by Eq.~\eqref{eq:ph180}
hence takes the form
$$
  \varphi^{\pm}_2(z)-\varphi^{\pm}_2(0)
    ={{A'_{\pm}C'_{\pm}}\over{B'_{\pm}}}z
       -{{(A'_{\pm}-B'_{\pm})C'_{\pm}}\over{B'_{\pm}D'_{\pm}}}
          \big[\Pi(D'_{\pm}z+K(\xi_{\pm}),B'_{\pm})-\Pi(K(\xi_{\pm}),B'_{\pm})\big],
  \eqdef{eq:ph210}
$$
with the coefficients $A'_{\pm}$, $B'_{\pm}$, $C'_{\pm}$ and $D'_{\pm}$ as previously
given by Eqs.~\eqref{eq:ph190}. To recapitulate the first-principles form of
this solution, just as previously for the pump wave, this is expressed as
$$
  \varphi^{\pm}_2(z)-\varphi^{\pm}_2(0)
    ={{A'_{\pm}C'_{\pm}}\over{B'_{\pm}}}z
       -{{(A'_{\pm}-B'_{\pm})C'_{\pm}}\over{B'_{\pm}D'_{\pm}}}
          \int^{D'_{\pm}z+K(\xi_{\pm})}_{K(\xi_{\pm})}
            {{du}\over{1-B_{\pm}\sn^2(u,\xi_{\pm})}}.
  \eqdef{eq:ph220}
$$

\centerline{\epsfxsize=150mm\epsfbox{figs/graph-06-pump.eps}}
\noindent{{\bf Figure~1.} Evolution of pump intensity
  ($S_0(z)/S_0(0)$), normalized ellipticity ($S_3(z)/S_0(z)$) and
  polarization state ($S_1(z)/S_0(z)$, $S_2(z)/S_0(z)$, $S_3(z)/S_0(z)$).
  The parameters used are as follows: Initial signal-to-pump ratio
  $s^2_+=s^2_-0.1$, normalized electric dipolar phase mismatch
  $\Delta k L=2.0$, a relative gyrotropic correction
  $\Delta\alpha/\Delta k=2.0$, and the LCP/RCP gain coefficients
  $\zeta_+=\kappa_+L(u^-_3(0))^{1/2}=4.0$ and
  $\zeta_-=\kappa_-L(u^+_3(0))^{1/2}=4.0$.}
\vfill\eject

\centerline{\epsfxsize=150mm\epsfbox{figs/graph-06-signal.eps}}
\noindent{{\bf Figure~2.} Evolution of signal intensity
  ($S_0(z)/S_0(0)$), normalized ellipticity ($S_3(z)/S_0(z)$) and
  polarization state ($S_1(z)/S_0(z)$, $S_2(z)/S_0(z)$, $S_3(z)/S_0(z)$),
  using the same parameters as in Fig.~1.}
\vfill\eject

\centerline{\epsfxsize=86mm\epsfbox{figs/graph-06-stokes-pump.eps}}
\noindent{{\bf Figure~3.} The polarization state of the pump wave mapped
  onto the Poincar\'e sphere, using the same parameters as in previous
  figures.}

\centerline{\epsfxsize=86mm\epsfbox{figs/graph-06-stokes-signal.eps}}
\noindent{{\bf Figure~4.} The polarization state of the signal wave mapped
  onto the Poincar\'e sphere, using the same parameters as in previous
  figures.}
\vfill\eject

\bye
